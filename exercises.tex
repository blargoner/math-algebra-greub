% Notes and exercises from Linear Algebra by Greub
% By John Peloquin
\documentclass[letterpaper,12pt]{article}
\usepackage{amsmath,amssymb,amsthm,enumitem,fourier,diagrams}

\newcommand{\R}{\mathbb{R}}

\newcommand{\iso}{\cong}

\newcommand{\after}{\circ}

\newcommand{\pair}[2]{\langle#1,#2\rangle}

\newarrow{Dashto}{}{dash}{}{dash}>

% Theorems
\theoremstyle{definition}
\newtheorem*{exer}{Exercise}

\theoremstyle{remark}
\newtheorem*{rmk}{Remark}

% Meta
\title{Notes and exercises from\\\textit{Linear Algebra}}
\author{John Peloquin}
\date{}

\begin{document}
\maketitle

\section*{Introduction}
This document contains notes and exercises from~\cite{greub}. Unless otherwise stated, \(\Gamma\)~denotes a field of scalars.

\section*{Chapter~I}
\subsection*{\S~1}
\begin{rmk}
The free vector space~\(C(X)\) is intuitively the space of all ``formal linear combinations'' of \(x\in X\).
\end{rmk}

\subsection*{\S~2}
\begin{exer}[5 - Universal property of~\(C(X)\)]
Let \(X\)~be a set and \(C(X)\)~the free vector space on~\(X\) (\S~1.7). Recall 
\[C(X)=\{\,f:X\to\Gamma\mid f(x)=0\text{ for all but finitely many }x\in X\,\}\]
The inclusion map \(i_X:X\to C(X)\) is defined by \(a\mapsto f_a\) where \(f_a\)~is the ``characteristic function'' of~\(a\): \(f_a(a)=1\) and \(f_a(x)=0\) for all \(x\ne a\). For \(f\in C(X)\), \(f=\sum_{a\in X}f(a)f_a\).
\begin{enumerate}
\item[(i)] If \(F\)~is a vector space and \(f:X\to F\), there is a unique \emph{linear} \(\varphi:C(X)\to F\) ``extending~\(f\)'' in the sense that \(\varphi\after i_X=f\):
\begin{diagram}[nohug]
X	&\rTo^{i_X}	&C(X)\\
	&\rdTo_f	&\dDashto>{\varphi}\\
	&			&F
\end{diagram}

\item[(ii)] If \(\alpha:X\to Y\), there is a unique \emph{linear} \(\alpha_*:C(X)\to C(Y)\) which makes the following diagram commute:
\begin{diagram}
X			&\rTo^{\alpha}			&Y\\
\dTo<{i_X}	&						&\dTo>{i_Y}\\
C(X)		&\rDashto_{\alpha_*}	&C(Y)
\end{diagram}
If \(\beta:Y\to Z\), then \((\beta\after\alpha)_*=\beta_*\after\alpha_*\).

\item[(iii)] If \(E\)~is a vector space, there is a unique linear map \(\pi_E:C(E)\to E\) such that \(\pi_E\after i_E=\iota_E\) (where \(\iota_E:E\to E\) is the identity map):
\begin{diagram}[nohug]
E	&\rTo^{i_E}			&C(E)\\
	&\rdTo_{\iota_E}	&\dDashto>{\pi_E}\\
	&			&E
\end{diagram}

\item[(iv)] If \(E\) and~\(F\) are vector spaces and \(\varphi:E\to F\), then \(\varphi\)~is linear if and only if \(\pi_F\after\varphi_*=\varphi\after\pi_E\):
\begin{diagram}[nohug]
E			&				&\rTo^{\varphi}		&				&F				&				&\\
			&\rdTo			&					&				&\vLine>{i_F}	&\rdTo			&\\
\dTo<{i_E}	&				&E					&\rTo^{\varphi}	&\HonV			&				&F\\
			&\ruTo>{\pi_E}	&					&				&\dTo			&\ruTo>{\pi_F}	&\\
C(E)		&				&\rTo_{\varphi_*}	&				&C(F)			&				&
\end{diagram}

\item[(v)] Let \(E\)~be a vector space and \(N(E)\)~the subspace of~\(C(E)\) generated by all elements of the form
\[f_{\lambda a+\mu b}-\lambda f_a-\mu f_b\qquad(a,b\in E\text{ and }\lambda,\mu\in\Gamma)\]
Then \(\ker\pi_E=N(E)\).
\end{enumerate}

\begin{proof}\
\begin{enumerate}
\item[(i)] By Proposition~II, since \(i_X(X)\)~is a basis of~\(C(X)\).

\item[(ii)] By~(i), applied to~\(i_Y\after\alpha\). Note \(\beta_*\after\alpha_*\)~is linear such that
\[(\beta_*\after\alpha_*)\after i_X=i_Z\after(\beta\after\alpha)\]
so \(\beta_*\after\alpha_*=(\beta\after\alpha)_*\) by uniqueness:
\begin{diagram}
X			&\rTo^{\alpha}		&Y			&\rTo^{\beta}	&Z\\
\dTo<{i_X}	&					&\dTo>{i_Y}	&				&\dTo>{i_Z}\\
C(X)		&\rTo_{\alpha_*}	&C(Y)		&\rTo_{\beta_*}	&C(Z)
\end{diagram}

\item[(iii)] By~(i), applied to~\(\iota_E\).

\item[(iv)] If \(\varphi\)~is linear, then \(\varphi\after\pi_E:C(E)\to F\) is linear and extends~\(\varphi\) in the sense that \(\varphi\after\pi_E\after i_E=\varphi\after\iota_E=\varphi\). However, \(\pi_F\after\varphi_*:C(E)\to F\) is also linear and extends~\(\varphi\) since
\[\pi_F\after\varphi_*\after i_E=\pi_F\after i_F\after\varphi=\iota_F\after\varphi=\varphi\]
By uniqueness, these two maps must be equal. Conversely, if these two maps are equal, then \(\varphi\)~is linear since \(\pi_F\after\varphi_*\)~is linear and \(\pi_E\)~is surjective.

\item[(v)] By~(iii),
\begin{align*}
\pi_E(f_{\lambda a+\mu b}-\lambda f_a-\mu f_b)&=\pi_E(f_{\lambda a+\mu b})-\lambda\pi_E(f_a)-\mu\pi_E(f_b)\\
	&=\lambda a+\mu b-\lambda a-\mu b\\
	&=0
\end{align*}
for all \(a,b\in E\) and \(\lambda,\mu\in\Gamma\). It follows that \(N(E)\subseteq\ker\pi_E\) since \(N(E)\)~is the \emph{smallest} subspace containing these elements and \(\ker\pi_E\)~is a subspace.

On the other hand, it follows from the fact that \(N(E)\)~is a subspace that
\[\sum\lambda_i f_{a_i}-f_{\;\sum\lambda_i a_i}\in N(E)\]
for all (finite) linear combinations. Now if \(g=\sum_{a\in E}g(a)f_a\in\ker\pi_E\), then
\[0=\pi_E(g)=\sum_{a\in E}g(a)\pi_E(f_a)=\sum_{a\in E}g(a)a\]
This implies \(f_{\;\sum_{a\in E}g(a)a}=f_0\in N(E)\). But by the above, \(g-f_0\in N(E)\), so \(g\in N(E)\). Therefore also \(\ker\pi_E\subseteq N(E)\).\qedhere
\end{enumerate}
\end{proof}
\begin{rmk}
Note (i)~shows that \(C(X)\)~is a universal (initial) object in the category of ``vector spaces with maps of~\(X\) into them''. In this category, the objects are maps \(X\to F\), for vector spaces~\(F\), and the arrows are \emph{linear} (i.e. structure-preserving) maps \(F\to G\) between the vector spaces which respect the mappings of~\(X\):
\begin{diagram}[nohug]
X	&\rTo	&F\\
	&\rdTo	&\dTo\\
	&		&G
\end{diagram}
By~(i), every object \(X\to F\) in this category can be obtained from the inclusion map \(X\to C(X)\) in a unique way. This is why \(C(X)\) is called ``universal''. This is only possible because \(C(X)\)~is free from any nontrivial relations among the elements of~\(X\), so any relations among the images of those elements in~\(F\) can be obtained starting from~\(C(X)\). This is why \(C(X)\)~is called ``free''. It is immediate from the universal property that \(C(X)\)~is unique up to isomorphism: if \(X\to U\) is also universal, then the composites \(\psi\after\varphi\) and~\(\varphi\after\psi\) of the induced linear maps \(\varphi:C(X)\to U\) and \(\psi:U\to C(X)\) are linear and extend the inclusion maps, so must be the identity maps on \(C(X)\) and~\(U\) by uniqueness; that is, \(\varphi\) and~\(\psi\) are mutually inverse and hence \emph{isomorphisms}. In fact they are also unique by the universal property.

Now (ii)~shows that we have a \emph{functor} from the category of sets into this category, which sends sets \(X\) and~\(Y\) to the objects \(X\to C(X)\) and \(Y\to C(Y)\), and which sends a set map \(\alpha:X\to Y\) to the linear map \(\alpha_*:C(X)\to C(Y)\). The functor preserves the category structure of composites of arrows.

In~(iii), we are ``forgetting'' the linear structure of~\(E\) when forming~\(C(E)\). For example, if \(E=\R^2\), then \(\pair{1}{1}=\pair{1}{0}+\pair{0}{1}\) in~\(E\), but \emph{not} in~\(C(E)\). The ``formal'' linear combination
\[\pair{1}{1}-\pair{1}{0}-\pair{0}{1}\]
is not zero in~\(C(E)\) because the pairs are unrelated elements (symbols) which are \emph{linearly independent}. Note \(\pi_E\)~is surjective (since \(\iota_E\)~is), so \(E\)~is a projection of~\(C(E)\). In~(iv), we see that \(\varphi:E\to F\) is linear if and only if it is a ``projection'' of \(\varphi_*:C(E)\to C(F)\).

In~(v), we see that \(\pi_E\)~just recalls the linear structure of~\(E\) that was forgotten in~\(C(E)\). In particular, \(C(E)/N(E)\iso E\). In other words, if you start with~\(E\), then forget about its linear structure, then recall that linear structure, you just get \(E\)~again.
\end{rmk}
\end{exer}

% References
\begin{thebibliography}{0}
\bibitem{greub} Greub, W. \textit{Linear Algebra}, 4th~ed. Springer, 1975.
\end{thebibliography}
\end{document}